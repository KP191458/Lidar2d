\section {Implementacja programowa}

\subsection {Użyte języki programowania i biblioteki}
Implementacja składała się z dwóch programów: Pierwszy z nich został napisany w jezyku programowania dla mikrokontrolerów Arduino (opartym na środowisku Wiring i zasadniczo na języku C/C++ \cite{arduino}) i służył do odczytu danych z dalmierza oraz sterowania silnikiem. Drugi został napisany w języku Python w wersji 3.9.5. i służył do generowania obrazu z odczytanych wcześniej punktów.\\

W programie napisanym w środowisku Arduino wykorzystaliśmy biblioteki:

\begin{enumerate}
    \item \textbf{AccelStepper 1.61.0} - biblioteka do sterowania silnikiem.
    \item \textbf{TFMini 1.01} - bilioteka do odczytu danych z dalmierzy serii TF.
\end{enumerate}

Napisany w języku Python program do generowania obrazu z przekazanych punktów wykorzystuje następujące biblioteki:

\begin{enumerate}
    \item \textbf{PyQt5}
    \item \textbf{pyqtgraph}
    \item \textbf{pyserial}
\end{enumerate}

Dwie pierwsze to biblioteka Qt służąca do budowy interfejsów graficznych i jej moduł do wykresów. Ostatnia to biblioteka do odczytu danych z portów szeregowych Arduino.\\

W pierwotnej koncepcji oba programy miały być napisane w języku C++. Problemem okazało się nieprzystosowanie najpopularniejszych bibliotek (m.in. gnuplot) do pracy z dynamicznymi danymi. Kolejna koncepcja poddawała pod rozwagę użycie czysto obiektowych języków Java bądź C\#. Ze względu na dynamiczny charakter pracy urządzenia i istotną dla niego wysoką szybkość przesyłu danych z dalmierza istniało podejrzenie że Java może mieć problem z rysowaniem tak dużej ilości wykresów w tak któtkim czasie. Implementacje rysowania wykresów dynamicznych pod swoją powłoką rysowały dla nowych danych kolejny raz cały wykres i podmieniały wygenerowany rysunek.\\

Optymalny okazał się język Python. Użyta biblioteka również rysowała za każdym razem cały wykres od nowa ale jej optymalizacja umożliwiała płynną obsługę napływających danych. Język Python ma również wbudowane funkcje parsowania tekstu i prostą obsługę tablic.

\subsection {Podstawy prawne użytego oprogramowania}
Arduino korzysta z licencji \textbf{LGPL} \cite{arduino_lic}, a użyte przeze mnie biblioteki
AccelStepper z \textbf{GPL} \cite{accelstepper_lic} a TFMini z \textbf{Public Domain} \cite{tfmini_lic}.\\

Python rozwijany jest jako projekt Open Source zarządzany przez Python Software Foundation, która jest organizacją non-profit. Jest udostępniony na licencji kompatybilnej z \textbf{GPL} \cite{python_lic}. PyQt5 jest na licencji \textbf{GPL v3.} \cite{pyqt5_lic}. PyQtGraph jest na licencji \textbf{MIT open-source license}. \cite{pyqtgraph_lic}\\

Niniejsza praca nie jest wykorzystaniem komercyjnym więc żadna z licencji nie zastała naruszona.

\subsection {Analiza wybranych fragmentów kodu}

Poniżej analiza najważniejszych elementów kodu źrodłowego które złożyły się na implementacje od strony oprogramowania.\\

\textbf{Odczyt danych z dalmierza}

\begin{lstlisting}[language=C++, caption=Odczyt danych z dalmierza]
    void loop() { 
        if (Serial1.available()) {  //check if serial port has data input
          if(Serial1.read() == HEADER) {  //assess data package frame header 0x59
            uart[0]=HEADER;
            if (Serial1.read() == HEADER) { //assess data package frame header 0x59
              uart[1] = HEADER;
              for (i = 2; i < 9; i++) { //save data in array
                uart[i] = Serial1.read();
              }
              check = uart[0] + uart[1] + uart[2] + uart[3] + uart[4] + uart[5] + uart[6] + uart[7];
              if (uart[8] == (check & 0xff)){ //verify the received data as per protocol
                dist = uart[2] + uart[3] * 256;     //calculate distance value
                strength = uart[4] + uart[5] * 256; //calculate signal strength value
                temprature = uart[6] + uart[7] *256;//calculate chip temprature
                temprature = temprature/8 - 256;
                Serial.print(step);
                Serial.print(" ");
                Serial.print("dist = ");
                Serial.print(dist); //output measure distance value of LiDAR
                Serial.print('\t');
                Serial.print("strength = ");
                Serial.print(strength); //output signal strength value
                Serial.print("\t Chip Temprature = ");
                Serial.print(temprature);
                Serial.println(" celcius degree"); //output chip temperature of Lidar
                take_step();         
              }
            }
          }
        }
      }
\end{lstlisting}
Powyższy kod jest kopią kodu z przykladów z biblioteki urządzeń TFMini. Kod działa w pętli ciągłej. Kod sprawdza odczytany na początku nagłówek z danych z portu. Jeśli nagłówek jest prawidłowy odczytane dalej wartość są umieszczane w tablicy typu int. Kolejnym krokiem jest konwersja odczytanych danych na porządane typy i przypisanie ich do zmiennych (zmienne oraz ich nazwy zostały zainicjowane wcześniej). Następnym krokiem jest wypisanie odczytanych danych na port wyjściowy urządzenia. Implementacja została uzupełniona o ostatni krok wywołanie funkcji take\_step() która obraca silnik krokowy o kolejny krok. Kod fukcji został opisany poniżej.\\

\textbf{Przesunięcie osi silnika o kolejny krok}

\begin{lstlisting}[language=C++, caption=Przesunięcie o kolejny krok]
void take_step(){
  if(dir==0)
    step++;
  else
    step--;

  if(step==0)
    dir=0;
  if(step==400)
    dir=1;

  myStepper.moveTo(step);
  delay(100);
  myStepper.run();
}
\end{lstlisting}
Funkcja zawiera proste instrukcje warunkowe które sprawdzają który krok został ostatnio podjęty (według zainicjowanego globalnie licznika kroków) oraz kierunek przesuwu. Przesów silnika odbywa się za pomocą rekomendowanej dla silnika biblioteki. Funkcja ta umożliwiła optymalny - wahadłowy - ruch osi silnika.\\

\textbf{Rysowanie punktów obrazu}

\begin{lstlisting}[language=Python, caption=Rysowanie obrazu]
    def update_plot_data(self):

        m = [ser.readline(100).decode('utf-8')]
        n = m[0].split()
        print(n)
        i = int(n[0])
        x_value = float(int(n[3])*math.sin(int(n[0])*math.pi/200.0))
        y_value = float(int(n[3])*math.cos(int(n[0])*math.pi/200.0))

        self.x.pop(i)
        self.x.insert(i, -x_value)

        self.y.pop(i)
        self.y.insert(i, y_value)

        self.data_line.setData(self.x, self.y)  # Update the data.
\end{lstlisting}

Funkcja ta odczytuje dane z portu szeregowego Arduino (całe linijki tekstu) a następnie dane umieszcza w tablicy danych typu string (funkcja split()). Natępnie używamy z tej tabeli jedynie aktualnego kąt (w gradach) i odległości (odczytanej z dalmierza). Dane te umożliwiają wylicznie współrzędnych rysowanych punktów (x\_value i y\_value). Kolejno punkty te zostają wrzucone w odpowiednie miejsca w kontener pozostałych punktów a na koniec obraz zostaje ponownie wygenerowany w celu aktualizacji o odczytany punkt.