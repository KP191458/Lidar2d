\section {Wnioski}
\subsection {Wnioski końcowe}
Pomimo czysto amatorskiego charakteru urządzenia osiągnięta dokładność była zaskakująco wysoka. Badane pomieszczenia odwzorowane zostały z dużą dokładnością a ich charakter, kształt i elementy charakterystyczne były wyraźnie rozpoznawalne. Największe nieścisłości pojawiały się w przypadku refleksyjnych powierzchni (szczególnie szklanych i metalowych).\\

Dokładność generowanego pomieszcze zależna była w znacznym stopniu od występujących w nim powierzchni oraz stopniu skomplikowania jego układu architektonicznego.\\

Ze względu na długi czas generowania obrazu (ok. jednej minuty), brak jednoznacznego układu odniesienia (np. według geograficznych stron świata) oraz wymienione wyżej przekłamania urządzenie może mieć jedynie charakter amatorski i jedynym jego zostosowaniem może być zastosowanie dydatktyczne.