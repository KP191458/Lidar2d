\section {Napotkane problemy}
\subsection {Synchronizacja odczytu danych z dalmierza}
Pierwotnym założeniem było aby ruch silnika krokowego był funkcją nadrzędną a odczyt odległości z dalmierza odbywał się tak jak nakazuje kod sterujący obrotem silnika. W takiej implementacji ok 70 \% odczytów odległości dawała puste odczyty. Zastosowanie opóżnienia nie przynosiło poprawy. Prostym rozwiązaniem okazało się uzależnienie obrotu silnika od odczytu odległości - innymi słowy: dopiero gdy odczytaliśmy pomiar ogledłości z dalmierza nadawaliśmy do silnika sygnał przemieszczenia o kolejny krok (z zastosowaniem dla pewności niewielkiego opóźnienia).\\

Najistotniejszym skutkiem uzależnienia przesówu silnika od odczytu dalmierza był brak możliwości dokładnego wysterowania prędkością obrotu silnika. Jedyna możliwość to zastosowanie dłuższego opóźnienia co skutkowało tym że czas wykonania pełnego obrotu urządzenia mogliśmy jedynie wydłużyć. 

\subsection {Refleksyjne powierzchnie}
Nawet lekki połysk powierzchni (np. kran czy lodówka) wprowadzały duże rozbierzności pomiędzy mierzoną a faktyczną odległością. W przypadku większego stopnia refleksyjności i nieregularnych krztałtów - jak np. wyżej wspomniany kran kuchenny - musiałem uciec się do prób okrycia jego powierzchni materiałami o nierefleksyjnej powierzchni.

\subsection {Okna}
Okna były kolejnym problematycznym elementem analizowanych pomieszczeń. Refleksyjna powierzchnia szyb dawała odczyty przekłamane w porównianiu z prawdziwą odległością. Co więcej elementy za oknem były niekiedy daleko poza zakresem w jakich pracuje dalmierz (12 metrów).\\

Skutkiem istnienia okien w badanych pomieszczeniach były duże przekłamania w odczytywanych odległościach a co za tym idzie bardzo duże przekłamania w tworzonym obrazie. Prostym rozwiązaniem okazało się użycie zaluzji okiennych które dostępne były w  badanych pomieszczeniach. Pozwoliło to na wygenerowanie precyzyjnego obrazu wewnętrznego obrysu pomieszczenia.

\subsection {Błędne odczyty dalmierza}
Raz na kilka tysięcy odczytów dalmierz pokazywał zupełnie błędną wartość - znacznie z poza zakresu. Pojawienie się takiej wielkości pogarszało znacznie kształt uzyskanego obrazu. Co więcej jeśli na obszarze rysowania mieliśmy włączone automatyczne skalowanie to wówczas wyskalowanie rysunku do rozmairu okna powodowało że prawidłowo rysowane pomieszsczenie było odzwierciedlane w zbyt małej skali.

\subsection {Czas wykonania pełnego obrazu}
Do wykonania pełnego obrazu potrzebowaliśmy 400 kroków. Ze względu na czas wykonania pojedyńczego pomiaru przez dalmierz oraz przesunięcia pozycji silnika czas trwania wykonania pełnego obrazu był znacznie wyższy od profesjonalnych urządzeń pomiarowych tego typu.

\subsection {Brak układu współrzędnych dla płaszczyzny obrotu silnika}
Tarcza obrotu silnika nie posiada układu współrzędnych - początkowego kąta obrotu. Korelacja pomiędzy początkowym kątem obrotu a porządanym miejscem w układzie współrzędnych na wygenerowanym obrazie musiała być ustawiona ręcznie. Włącznie zasilania silnika musiało zbiedz się z momentem w którym program rozpoczynał generowanie obrazu.\\

Problem wynikał z natury urządzeń jakimi są silniki krokowe - każdy krok jest sterowany za pomocą wygenerowania przez sterownik różnicy potencjałów. Różnica ta jest charaktrystyczna dla silnika i powoduje przemieszczenie części obrotowej silnika o jeden krok. Istotą tych urządzeń jest właśnie obrót o dany krok (zgodnie lub przeciwnie do ruchu wskazówek zegara). Osoba korzystająca ze sterownika posługuje się głównie ilością i kierunkiem kroków a co za tym idzie urządzenia te nie zapewniają układu współrzędnych i przemieszczeń o zadane kąty.